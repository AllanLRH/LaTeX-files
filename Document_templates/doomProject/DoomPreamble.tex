%%%%%%%%%%%%%%%%%% PREAMBLE BEGIN %%%%%%%%%%%%%%%%%%
% A templet for mathematical LaTeX2e documents - partially based on the preambles avaible at http://psi.nbi.dk/groups/psi/wiki/Preamble


%%%%%%%%%%%%%%%%%% ESSENTIAL FORMATTING %%%%%%%%%%%%%%%%%%
\documentclass[a4paper,11pt]{scrartcl}
% \documentclass[a4paper,11pt,notitlepage]{report}

% If activated, labels will be shown in the margin. Use only when in draft mode.
%\usepackage[notref,notcite]{showkeys}



%%%%%%%%%%%%%%%%%% FOR DANISH DOCUMENTS %%%%%%%%%%%%%%%%%%

% Chapters is given danish names and so on.
% \usepackage[danish]{babel}
\usepackage[english]{babel}

% Makes commas work as decimal-points (fixes the spacing around.
% \usepackage{icomma}

% Sets the indentation for new paragraphs. Per default LaTeX creates some indentation whenever a new paragraph begins. While this might look visually pleasing, it's an english/american convention, and not "correct" danish typestting.
% \parindent=0pt

 % For use on Unix systems (Mac/Linux). Use this or the one below.
\usepackage[utf8]{inputenc}

% For use on Windows systems. Use this or the one above.
% \usepackage[T1]{fontenc}



%%%%%%%%%%%%%%%%%% ESSENTIAL PACKAGES %%%%%%%%%%%%%%%%%%
% Make PDF-files searchable
\usepackage{cmap}


% Better mathematics, customization of equation numbering and much much more
% Documentation: http://mirrors.ctan.org/macros/latex/contrib/mathtools/mathtools.pdf
\usepackage{mathtools}
% Defines the new size of tags
\newtagform{footnotetext}[]{\footnotesize(}{)}
% Initialize the new tag size defined above
\usetagform{footnotetext}
% Hides numbers on all equations with no refferences.
\mathtoolsset{showonlyrefs}

% \makeindex generates index, required for the TOC. Remember to compile two times to ensure that all changes and references have been updated properly!
\usepackage{makeidx}

% Typesets the pagenumber of the last page by entering \pageref{Lastpage}
\usepackage{lastpage}

% For including figures
\usepackage{graphicx}



%%%%%%%%%%%%%%%%%% ADDITIONAL FORMATTING %%%%%%%%%%%%%%%%%%
% Adds space between lines. The default linespace is multiplied by the argument
% \linespread{1.15}

% % Sets the indentationlength for new equations.
% \setlength{\mathindent}{2.4cm}

% Overfull hbox under 3pt ignoreres - makes LaTeX a but less fussy, but might also make things look a little bit less pleasing to the eye.
\hfuzz=3pt


%     *** Explanation of the enumerations ***
% Command                         Level             Comment
%
% \part{part}                      -1             not in letters
% \chapter{chapter}                 0             only books and reports
% \section{section}                 1             not in letters
% \subsection{subsection}           2             not in letters
% \subsubsection{subsubsection}     3             not in letters
% \paragraph{paragraph}             4             not in letters
% \subparagraph{subparagraph}       5             not in letters
%
% Source: https://secure.wikimedia.org/wikibooks/en/wiki/LaTeX/Document_Structure#Sectioning_Commands

% Sets the maximal "depth" of the TOC to 3.
\setcounter{tocdepth}{3}

% Sets the depth of numbered 3, the default value is 2.
\setcounter{secnumdepth}{2}

% Start counting from 1, because we don't want the first page after the front page 1.
\setcounter{page}{0}
\setcounter{section}{0}

% Sets rules for equation numbering to follow a section or subsection ( replace the "section" -argument with subsection ).
% Default values set gives equation 3 in section 1 the number (tag) (1.3)
% Docuentation: http://en.wikibooks.org/wiki/LaTeX/Tips_and_Tricks#Grouping_Figure.2FEquation_Numbering_by_Section
\numberwithin{equation}{section}

% This package allows you to custimize your titles, subtitle, subsubtitles and so forth. Only works with some documentstyles, Report and Article included.
% Current setting set all titles to be typeset in a sans-serif font.
% Unlike when you format the title manually, the formatting is not duplicated in the TOC. Documentation: http://www.ctex.org/documents/packages/layout/sectsty.pdf
% \usepackage{sectsty}
%     \allsectionsfont{\sffamily}

% Use Asteriks, dagger, ddagger... for footnotes
\renewcommand*{\thefootnote}{\fnsymbol{footnote}}

% The geometry package allows for customisation of margins
% Small tutorial: https://www.sharelatex.com/learn/Page_size_and_margins
% Through tutorial: https://en.wikibooks.org/wiki/LaTeX/Page_Layout
\usepackage[a4paper]{geometry}

% It's considered obsolete, but it's very explicit and easy to use, which makes it a good alternative for geometry for a quick fix
% Options are: \marginsize{left}{right}{top}{bottom}
% \usepackage{anysize}
%    \marginsize{2.0cm}{2.5cm}{2.2cm}{3.2cm}


%    ***    Font options    ***
% All fonts include mathematical characters for equations, and support the danish letters æøåÆØÅ.
% Documentation: http://www.tug.dk/FontCatalogue/mathfonts.html
% Remove the comment sign to use a font.

% Palatino
%\usepackage{mathpazo}

% Times
%\usepackage{mathptmx}

% Adobe Utopia. Preview: http://www.tug.dk/FontCatalogue/utopia-mathdesign/
\usepackage[adobe-utopia]{mathdesign}

% Source Code Pro for typewriter text
% \usepackage{sourcecodepro}

% Anonymous Pro for typewriter text  NOTE: Documentation states that \usepackage[T1]{fontenc} and \usepackage{textcomp} must be loaded:
% ftp://ftp.dante.de/tex-archive/fonts/anonymouspro/doc/AnonymousPro.pdf
\usepackage[scale=0.95,ttdefault]{AnonymousPro}

% Typeset symbols like \Letter (for email) and \Telefon (for phone number)
\usepackage{marvosym}

% Arev    (Sans-Serif)
%\usepackage{arev}



%   ***   Fancyheaders:   ***

% The Fancyheaders-package makes those fancy header and footer seen in a lot af LaTeX-documents.
\usepackage{fancyhdr}
\pagestyle{fancy}                                                % Initialises layout options for the package
\fancyhead{}                  % Vipes all header-fields
\fancyhead[L]{My Title}
\fancyhead[R]{My Subtitle}
\headheight 14.5pt            % Sets the height of the header-field.
\fancyfoot{}                  % Vipes all footer-fields
\fancyfoot[C]{Page \thepage{} of \pageref{LastPage}}
\renewcommand{\headrulewidth}{0.2pt}  % Set thickness of the headers line/ruler
\renewcommand{\footrulewidth}{0.0pt}  % Set to 0 (zero) to remove footer rule

% Ensures that first page (front page) have a clean header
\fancypagestyle{firstPageStyle}
{
   \fancyhead{}
   \fancyfoot{}
   \headheight 0pt
   \renewcommand{\headrulewidth}{0.0pt}
   % \fancyfoot[C]{Page \thepage{} of \pageref{LastPage}}
}



%%%%%%%%%%%%%%%%%% ADDITIONAL PACKAGES %%%%%%%%%%%%%%%%%%
% Greatly improves on PDFLaTeX's typesetting
\usepackage{microtype}

% Make it possible to color selected rows of a table, og to set up a color-pattern for the table.
% Documentation: http://en.wikibooks.org/wiki/LaTeX/Tables#Alternate_Row_Colors_in_Tables
\usepackage[table,dvipsnames]{xcolor}

% Inline enumerations via the inparaenum environment
% Examples on blog: https://texblog.org/2013/02/01/inline-lists-in-latex-using-paralist/
% Documentation: http://mirrors.ctan.org/macros/latex/contrib/paralist/paralist.pdf
\usepackage{paralist}

% Ensures that the TOC is not present in the TOC, and that the list of litterature is included in the TOC.
\usepackage[nottoc]{tocbibind}

% Makes the TOC "clickable" and more. Documentation: https://secure.wikimedia.org/wikibooks/en/wiki/LaTeX/Hyperlinks#Customization
\usepackage[colorlinks=true,linkcolor=black,urlcolor=cyan]{hyperref}

% Typeset urls, and emails with properly formatted @'s
% Use like: \url{ }, \url| | or \xyz
% Documentation: http://mirrors.dotsrc.org/ctan/macros/latex/contrib/url/url.pdf
\usepackage{url}
% Define the \email-function
\DeclareUrlCommand\email{\urlstyle{rm}}

% Include source code formatted by Pygments
% Tutorial: https://www.sharelatex.com/learn/Code_Highlighting_with_minted
%
% Inline typesetting:
%        \mint{html}|<h2>Something <b>here</b></h2>|
%
% Include file:
%       \inputminted{octave}{BitXorMatrix.m}
%
%
% \begin{minted} {python}
% def sayHi(message=None):
%     if message:
%         print(message)
%     else:
%         print('Hi!')
% \end{minted}
%
\usepackage
[frame=lines,  % Draw lines over and under source code
framesep=2mm,
baselinestretch=1.2,
fontsize=\footnotesize,
linenos  % Show line numbers
]{minted}

% For creating enviroment with multiple colums.
% Note that figures is only supported in the \begin{figure*} -mode (*-mode), and they are stretched over all columns.
% Initiated as: \begin{multicols} ... \end{multicols}.
% Example: http://en.wikibooks.org/wiki/LaTeX/Page_Layout#Multi-column_pages
% Documentation: http://www.ctan.org/tex-archive/macros/latex/required/tools/multicol.pdf
\usepackage{multicol}
\setlength{\columnsep}{0.9cm}

% Allows printing of "nice" fractions like ½, a style often used when typesetting units. use as \nicefrac{1}{2} for typesetting ½ inside a mat enviroment.
\usepackage[nice]{nicefrac}

% Makes it possible to give the optional placeholder argument "H" for floats, forcing the float to be inserted immediately. This might create "holes" in the text!
\usepackage{float}

% Makes it possible to issue the command "\FloatBarrier" , "flushing" all floats in the buffer to the document.
% It does not create a new page like the "\clearpage".
% Also, the "[section]"-option ensures that floats is confined the the section where the insersion-code was issued (or right above or below - remove the "[above]" or "[below]" oprtions not allow this behavior).
\usepackage[section,above,below]{placeins}

% Typeset better tables, with proper vertical spacing, ofr instance.
% Tutorial: http://en.wikibooks.org/wiki/LaTeX/Tables#Professional_tables
% Documentation: http://mirrors.med.harvard.edu/ctan/macros/latex/contrib/booktabs/booktabs.pdf
\usepackage{booktabs}

% Color individual rows in the booktabs enviroment.
% Docuentation: http://ftp.math.purdue.edu/mirrors/ctan.org/macros/latex/contrib/colortbl/colortbl.pdf
\usepackage{colortbl}

% Lets one row expand over multiple colulms.
% Documentation: https://secure.wikimedia.org/wikibooks/en/wiki/LaTeX/Tables#Spanning
\usepackage{multirow}

% Allows custom formatting of the caption-text of floats.
% Documentation: http://mirrors.dotsrc.org/ctan/macros/latex/contrib/caption/caption-eng.pdf
\usepackage[labelfont=bf, font=small, justification=centerlast, margin=15pt, parskip=0pt]{caption}
\setlength{\abovecaptionskip}{-1ex}
% \setlength{\belowcaptionskip}{1ex}

% For printing figures and tables side by side: http://en.wikibooks.org/wiki/LaTeX/Floats,_Figures_and_Captions#Subfloats
\usepackage{subcaption}

% Inline figures: http://en.wikibooks.org/wiki/LaTeX/Floats,_Figures_and_Captions#Wrapping_text_around_figures
% Official documentation: http://mirrors.ctan.org/macros/latex/contrib/wrapfig/wrapfig-doc.pdf
\usepackage{wrapfig}

% Make cool'n fancy boxes is LaTeX.
% Documentation: http://mirrors.dotsrc.org/ctan/macros/latex/contrib/fancybox/fancybox-doc.pdf
\usepackage{fancybox}

% Make TODO-notes in boxes.
% Example:
%       \todo{My todo note}
% Documentation: http://mirrors.dotsrc.org/ctan/macros/latex/contrib/todonotes/todonotes.pdf
\usepackage[english]{todonotes}

% For typesetting units, for instance 50 N is typeset as \SI{50}{N}
\usepackage{siunitx}
\sisetup{
%    list-final-separator = {\translate{og} },
%    range-phrase = { \translate{til} },
    separate-uncertainty = true,
    multi-part-units = single,
    range-units = single,
    group-decimal-digits = false,
    exponent-product = \cdot,
    per-mode = fraction
}
% Declare unit px for pixel. Used with the siunitx package
\DeclareSIUnit{\pixels}{px}

% Convert EPS to PDF before compiling with PDFLATEX. Allows insertion of EPS-graphics.
\usepackage{epstopdf}

% By default, the package will generate a bit over one page, paragraph 1-7, using \lipsum or \lipsum[3-56] for multiple paragraphs.
\usepackage{lipsum}

%%%%%%%%%%%%%%%%%% USER DEFINED COMMANDS %%%%%%%%%%%%%%%%%%

%   ***   Physics Formatting   ***
% Activete here (bold, non-italicized unit vectors) or under the *** Mathematical Formatting ***
\renewcommand{\vec}[1]{\mathbf{#1}}
\newcommand{\vecc}[2]{\left(\begin{array}{c} #1 \\ #2 \end{array}\right)}
\newcommand{\veccc}[3]{\left(\begin{array}{c} #1 \\ #2 \\ #3 \end{array}\right)}
\newcommand{\I}{\mathbf{\widehat{i}}}
\newcommand{\J}{\mathbf{\widehat{j}}}
\newcommand{\K}{\mathbf{\widehat{k}}}
\newcommand{\erho}{{\widehat{\mathbf{e}}}_{\mathrm{\rho}}}
\newcommand{\ephi}{{\widehat{\mathbf{e}}}_{\mathrm{\Phi}}}
\newcommand{\ethe}{{\widehat{\mathbf{e}}}_{\mathrm{\theta}}}
\newcommand{\ez}{{{\widehat{\mathbf{e}}}_{z}}}
\newcommand{\er}{{{\widehat{\mathbf{e}}}_{r}}}
\newcommand{\es}{{{\widehat{\mathbf{e}}}_{s}}}
\newcommand{\uvec}[1]{\mathbf{\widehat{#1}}}  % Unit vector


%   ***   Mathematical Formatting   ***
% Activete here (slender, italicized unit vectors) or under the *** Physics formatting ***
%\newcommand{\I}{\mathrm{\widehat{i}}}
%\newcommand{\J}{\mathrm{\widehat{j}}}
%\newcommand{\K}{\mathrm{\widehat{k}}}
%\newcommand{\erho}{\mathrm{\widehat{e_{\rho}}}}
%\newcommand{\ephi}{\mathrm{\widehat{e_{\Phi}}}}
%\newcommand{\ez}{\mathrm{\widehat{e_{z}}}}
%\newcommand{\er}{\mathrm{\widehat{e_{r}}}}
%\newcommand{\es}{\mathrm{\widehat{e_{s}}}}


% Some math operators (like \sin, \cos...)
\DeclareMathOperator*{\sinc}{sinc}
\DeclareMathOperator*{\fft}{fft \,}
\DeclareMathOperator*{\ifft}{ifft \,}
\DeclareMathOperator*{\ceil}{ceil \,}
\DeclareMathOperator*{\floor}{floor \,}
\DeclareMathOperator*{\round}{round \,}
\DeclareMathOperator*{\Tr}{Tr \,}      % Matrix trace
\newcommand{\T}{^{\mathrm{T}}}         % Matrix transpose

% Typesets a upright d.
% Intended for use in integrals like $ \int 2x dx $, which should be written as $ \int 2x \di x $.
% A little white space is added in front of the "d".
\newcommand{\di}{\; \mathrm{d}}

% Write "\p" instead of "\partial".
% Also, it adds a little white space in front of the "partial"-symbol.
\newcommand{\p}{\; \partial}

% Typesets a "d/d"-fraction, with optional arguments (placed behind the "d"'s in the nominator and denominator).
% Use as \dd{y}{x} for a "dy/dx"-fraction.
\newcommand{\dd}[2]{\frac{\mathrm{d} #1}{\mathrm{d} #2}}

% Typesets a "\partial/\partial"-fraction, with optional arguments (placed behind the "\partial"'s in the nominator and denominator).
% Use as \pd{y}{x} for a "\partial y / \partial x"-fraction.
\newcommand{\pd}[2]{\frac{\partial #1}{\partial #2}}

% A short, quicj way of inserting a 0.8 cm blank vertical space.
\newcommand{\vsp}{\vspace{0.8cm}}

% Typeset a non-italicized e in math enviroments, argument being the exponent.
% To typeset $ e^{x+y} $, use $ \e^{x+y} $, and the e wont be italicized.
\newcommand{\e}[1]{\mathrm{e}^{#1}}

% Makes it easy to reference to something severel pages away.
% Otherwise, it works as the "\ref"-command.
\newcommand{\fullref}[1]{\ref{#1} på side~\pageref{#1}}

% Lats you write an averaged function "<f(x)>" as $ \avg{f(x)} $
\newcommand{\avgg}[1]{\left\langle #1 \right\rangle}

% Shrinks the size of the argument, and puts it in math-enviroment.
% Well suited for small inline arrays.
\newcommand{\vecsm}[1]{\mbox{\footnotesize ${#1}$ }}

%    *** New Enviroments ***
% Define the color used as a background color
\definecolor{MyGray}{rgb}{0.83,0.83,0.83}
% Define the width of a borden -- only relevant if the "\fbox"-command uncommentet.
\setlength{\fboxrule}{0.5pt}

% Enviroment used as \graybox{  }, the contents of the box between the curly brackes.
% Contents may be broken into severel lines.
\newcommand{\graybox}[1]{\vspace*{-0.2cm}
                       \begin{center}
                        \colorbox{MyGray}{
                         %\fbox{
                          \begin{minipage}[t]{0.97\textwidth}
                            {\small #1 }
                          \end{minipage}
                         %}
                        }
                       \end{center}  }


%%%%%%%%%%%%%%%%%% DOCUMENTPART BEGIN %%%%%%%%%%%%%%%%%%
\makeindex
